\documentclass[12pt,a4paper]{article}
\usepackage[utf8]{inputenc}
\usepackage[usenames,dvipsnames]{xcolor}
\pagecolor{white}
\usepackage{graphicx}
\graphicspath{C:\Users\user\Desktop\Learning LaTeX}
\usepackage{amsmath,amssymb}
\usepackage{fancyvrb, fancyheadings}
\usepackage{fancyhdr}
\usepackage{tikz, tcolorbox,tikzcodeblocks}
\usepackage{pgfplots}
\pgfplotsset{compat=1.17}
\usepackage{halloweenmath}
\usepackage{enumerate}
%% Another packages
\usepackage{geometry}
\geometry{left=25mm, top=30mm}
\usepackage{physics,romannum}
\usepackage{import}
\usepackage{hyperref}
\hypersetup{
    colorlinks=true,
    linkcolor=cyan,
    filecolor=magenta,      
    urlcolor=blue
}
\usepackage{pdfpages,multicol,enumitem}







\title{{\bf \underline{ASSIGNMENT: MATHEMATICAL MODELING}}}
\author{\bf Akhlak Ansari}
%\date{\bf December 4, 2022}

\begin{document}
    \maketitle

    \pagestyle{fancy}
    \fancyfoot{}
    \color{black}
    %% \fancyhf{}
    \lhead{\includegraphics[scale=0.14]{DDU_Logo.png}}
    \rhead{Assignment : Mathematical Modeling}
    % \renewcommand{\headrulewidth}{0.5mm}
    %% \rfoot{\thepage}
    \lfoot{By: Akhlak Ansari}
    \rfoot{\thepage}
    \renewcommand{\footrulewidth}{1pt}

    \begin{center}
      
        \includegraphics[]{DDU_Logo.png}\\[3mm]
        \textbf{ {\LARGE Department of Mathematics and Statistics}}
       
        \vspace{7.8cm}

        \textbf{Session: 2022-23}

    \end{center}

    \vspace*{0.5cm}

    \begin{center}
        \section*{\underline{{\bf MCQ QUESTIONS WITH SOLUTIONS}}}
    \end{center}

    \begin{enumerate}
        \item Bacteria in a certain culture increase at a rate proportional to the number present.If the number of bacteria increases from $1000$ to $2000$ in one hour, then find the number of bacteria at the end of $6$ hours. 
        %\begin{multicols}{2}
        \begin{enumerate}
            \item $48000$
            \item $94000$
            \item $84000$
            \item \color{blue}{$64000$}
        \end{enumerate}
        %\end{multicols}
    \end{enumerate}

    \begin{itemize}
        \item \subsubsection*{\underline{{\bf Hints and Solution:}}}
    \end{itemize}

    Let $x(t)$ denotes the size(number) of the bacteria at any time $t$. Then, the given problem is governed by the differential equation,
    \begin{equation*}
        \frac{dx}{dt} = rx \tag*{(1)}
    \end{equation*}
    \[\mbox{Whose solution is given by,}\]
    
    \begin{equation}
         x(t) = x_0 e^{rt} \tag*{(2)}
    \end{equation}

    where $x_0 = 1000,$ when $t = 0$.

    Again, Since the number of bacteria becomes $2000$ at the end of one hour i.e. 

    $x(1) = 2000$ at $t = 1$, therefore by using the equation(2), we have 
    \[2000 = 1000 e^{r}\]
    which gives, 
    \begin{equation*}
        e^r = 2 \tag*{(3)}
    \end{equation*}

    Now, let $x(6)$ be the number of bacteria at the end of $6$ hours, therefore again by using equation(2), we have,
    \[x(6) = 1000 e^{r(6)} = 1000(2)^6\]
    \[\mbox{or},\ \ x(6) = 64000\]

    So, (d)  $\color{blue}{64000}$ is the correct option.

    \pagebreak

    \vspace*{0.5cm}

    \begin{enumerate}[label=2.]
        \item What are the behaviour of the curve in {\bf Exponential population growth} ?
        \begin{enumerate}
            \item It's J shaped curve
            \item It's S shaped curve
            \item It's a parabolic curve
            \item It's a hyperbolic curve
        \end{enumerate}
    \end{enumerate}

    Correct option is, (a) J shaped curve
    \vspace*{0.5cm}

    \begin{enumerate}[label=3.]
        \item What are the behaviour of the curve in {\bf Logistics population growth} ?
        \begin{enumerate}
            \item It's J shaped curve
            \item It's a hyperbolic curve
            \item It's a S shaped curve
            \item it's a Y shaped curve
        \end{enumerate}
    \end{enumerate}

    Correct option is, (c) S shaped curve
    \vspace*{0.5cm}

    \begin{enumerate}[label=4.]
        \item Which of these is not a {\bf greenhouse gas} ?
        \begin{enumerate}
            \item Oxygen
            \item Methane
            \item Corbon di-oxide
            \item Ozone
        \end{enumerate}
    \end{enumerate}

    Correct option is, (a) Oxygen
    \vspace*{0.5cm}

    \begin{enumerate}[label=5.]
        \item What do you mean by {\bf Emigration} ?
        \begin{enumerate}
            \item Population leaving the space 
            \item Population comes from outside the space
            \item Population died
            \item Population take birth
        \end{enumerate}
    \end{enumerate}

    Correct option is, (a) Population leaving the space

    \pagebreak

    \vspace*{0.5cm}

    \begin{enumerate}[label=6.]
        \item Which of the following counter acts biotic potential ?
        \begin{enumerate}
            \item Limitations of food supply
            \item Predation
            \item Competition
            \item All of the Above
        \end{enumerate}
    \end{enumerate}

    Correct option is, (d) All of the Above

    \vspace*{0.5cm}

    \begin{enumerate}[label=7.]
        \item Which of the these pairs are prey-predator model of Lotka-Volterra ?
        \begin{enumerate}
            \item $\frac{dx}{dt} = \alpha x + \beta xy$
            

            $\frac{dy}{dt} = \delta xy - \gamma y$
            \item $\frac{dx}{dt} = \alpha x + \beta xy$
            

            $\frac{dy}{dt} = \delta xy + \gamma y$
            \item $\frac{dx}{dt} = \alpha x - \beta xy$
            

            $\frac{dy}{dt} = \delta xy - \gamma y$
            \item $\frac{dx}{dt} = \alpha x - \beta xy$
            

            $\frac{dy}{dt} = \delta xy + \gamma y$
        \end{enumerate}
    \end{enumerate}

    Correct option is, (c) \begin{equation*}
        \begin{split}
            \frac{dx}{dt} & = \alpha x - \beta xy\\
            \frac{dy}{dt} & = \delta xy - \gamma y
        \end{split}
    \end{equation*}

    \vspace*{0.5cm}

    \begin{enumerate}[label=8.]
        \item Lotka-Volterra model is based on,
        \begin{enumerate}
            \item Logistics population growth
            \item Exponential population growth
            \item Both (a) and (b)
            \item None of the Above
        \end{enumerate}
    \end{enumerate}

    Correct option is, (a) Logistics population growth

    \pagebreak

    \vspace*{0.5cm}

    \begin{enumerate}[label=9.]
        \item In two species model, what's the meaning of prey and predator ?
        \begin{enumerate}
            \item Hunter organisms and hunted organisms
            \item Hunted organisms and  hunter organisms
            \item Organisms that are died and organisms that are take birth
            \item None of the Above
        \end{enumerate}
    \end{enumerate}

    Correct option is, (b) Hunted organisms and  hunter organisms

    \vspace*{0.5cm}

    \begin{enumerate}[label=10.]
        \item Lotka-Volterra prey-predator differential equations are, 
        \begin{enumerate}
            \item Non-linear differential equations
            \item Linear differential equations
            \item Ordinary differential equations
            \item None of the Above
        \end{enumerate}
    \end{enumerate}

    Correct option is, (a) Non-linear differential equations

    \vspace*{0.5cm}

    \begin{enumerate}[label=11.]
        \item What is {\bf Epidemic} ?
        \begin{enumerate}
            \item A large long-term outbreak of disease
            \item A small short-term outbreak of disease
            \item A large short-term outbreak of disease
            \item None of the Above
        \end{enumerate}
    \end{enumerate}

    Correct option is, (c) A large short-term outbreak of disease
    \vspace*{0.5cm}

    \begin{enumerate}[label=12.]
        \item In SI Model which is true, 
        \begin{enumerate}
            \item $\frac{dS}{dt}+\frac{dI}{dt}=0$
            \item $\frac{dS}{dt}-\frac{dI}{dt}=0$
            \item $\frac{dS}{dt}+\frac{dI}{dt}=1$
            \item None of the Above
        \end{enumerate}
    \end{enumerate}

    Correct option is, (a) $\frac{dS}{dt}+\frac{dI}{dt}=0$

    \pagebreak

    \vspace*{0.5cm}

    \begin{enumerate}[label=13.]
        \item Basic reproductive ratio, $R_0$ in SIR Model is given by, 
        \begin{enumerate}
            \item $\frac{\beta S_0}{\gamma}$
            \item $\frac{\gamma S_0}{\beta}$
            \item $\frac{\beta I_0}{\gamma}$
            \item None of the Above
        \end{enumerate}
    \end{enumerate}
    where, $\beta$ is the transmission contact rate and $\gamma$ is tne mean recovery rate.

    Correct option is, (a)  $\frac{\beta S_0}{\gamma}$

    \vspace*{0.5cm}

    \begin{enumerate}[label=14.]
        \item Most pandemics have arisen from influenza viruses from which of the following animals ?
        \begin{enumerate}
            \item Pigs
            \item Wild birds
            \item Humans
            \item Bats
        \end{enumerate}
    \end{enumerate}

    Correct option is, (b) Wild birds

    \vspace*{0.5cm}

    \begin{enumerate}[label=15.]
        \item SARS is described as a zoonotic-virus, what does this mean ?
        \begin{enumerate}
            \item Such viruses are confined to animals
            \item They do not cause disease in humans
            \item They cause pandemics
            \item They emerge from animals to cross the species barrier infrequently
        \end{enumerate}
    \end{enumerate}

    Correct option is, (d) They emerge from animals to cross the species barrier infrequently

    \vspace*{0.5cm}

    \begin{enumerate}[label=16.]
        \item The area under the serum concentration time curve of the drug represents :
        \begin{enumerate}
            \item The biological half life of the drug
            \item The amount of the drug in the original dosage from
            \item The amount of drug absorbed
            \item The amount of drug excreted in the urine
        \end{enumerate}
    \end{enumerate}

    Correct option is, (c) The amount of drug absorbed
    
    \pagebreak

    \vspace*{0.5cm}

    \begin{enumerate}[label=17.]
        \item Time dependent change in drug kinetics is known as:
        \begin{enumerate}
            \item Pharmacokinetics
            \item Chronokinetics
            \item Drug regulation
            \item None of the Above
        \end{enumerate}
    \end{enumerate}

    Correct option is, (b) Chronokinetics

    \vspace*{0.5cm}

    \begin{enumerate}[label=18.]
        \item Which of the following drugs get distributed to the same extent in both lean and adipose tissue:
        \begin{enumerate}
            \item Phenytoin
            \item Digoxin
            \item Antibiotics
            \item Caffeine
        \end{enumerate}
    \end{enumerate}

    Correct option is, (d) Caffeine

    \vspace*{0.5cm}

    \begin{enumerate}[label=19.]
        \item Which of the following is not a mechanism for pharmacokinetics analysis:
        \begin{enumerate}
            \item Compartment analysis
            \item Non-compartment analysis
            \item Physiologic modeling
            \item Human model
        \end{enumerate}
    \end{enumerate}

    Correct option is, (d) Human model

    \vspace*{0.5cm}

    \begin{enumerate}[label=20.]
        \item Which organs comprise the central compartment in a two compartment model:
        \begin{enumerate}
            \item Liver
            \item Muscles
            \item Adipose
            \item Skin
        \end{enumerate}
    \end{enumerate}

    Correct option is, (a) Liver

    \begin{itemize}
        \item For more info regarding this document visit: \url{https://github.com/akhlak919}
    \end{itemize}







    


    








    

\end{document}    