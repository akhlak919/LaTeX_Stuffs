\documentclass[12pt,a4paper]{article}
\usepackage[utf8]{inputenc}
\usepackage[usenames,dvipsnames]{xcolor}
\pagecolor{white}
\usepackage{graphicx}
\graphicspath{C:\Users\user\Desktop\Learning LaTeX}
\usepackage{amsmath,amssymb}
\usepackage{fancyvrb, fancyheadings}
\usepackage{fancyhdr}
\usepackage{tikz, tcolorbox,tikzcodeblocks}
\usepackage{pgfplots}
\pgfplotsset{compat=1.17}
\usepackage{halloweenmath}
\usepackage{enumerate}
%% Another packages
\usepackage{geometry}
\geometry{left=25mm, top=30mm}
\usepackage{physics,romannum}
\usepackage{import}
\usepackage{hyperref}
\hypersetup{
    colorlinks=true,
    urlcolor=blue
}







\title{{\bf \underline{ASSIGNMENT: RIGID BODY DYNAMICS}}}
\author{\bf Akhlak Ansari}
%\date{\bf December 4, 2022}

\begin{document}
    \maketitle

    \pagestyle{fancy}
    \fancyfoot{}
    \color{black}
    %% \fancyhf{}
    \lhead{\includegraphics[scale=0.14]{DDU_Logo.png}}
    \rhead{Assignment:Rigid Body Dynamics}
    % \renewcommand{\headrulewidth}{0.5mm}
    %% \rfoot{\thepage}
    \lfoot{By: Akhlak Ansari}
    \rfoot{\thepage}
    \renewcommand{\footrulewidth}{1pt}

    \begin{center}
      
        \includegraphics[]{DDU_Logo.png}\\[3mm]
        \textbf{ {\LARGE Department of Mathematics and Statistics}}
       
        \vspace{7.5cm}

        \textbf{Session: 2022-23}

    \end{center}
    
    \begin{center}
        \section*{\LARGE{\underline{Question}}}
    \end{center}

    A finite string has two masses $M$ and $M'$ tied to its end and passes over a rough pulley of mass $m$, whose center is fixed.Then prove that,
    \[T = T'e^{\mu \pi}\]
    where $T$ and $T'$, both are tension of string or rope according to motion and $\mu$ is the coefficient of acceleration.

    \vspace*{1cm}

  
    \section*{\LARGE{\underline{Solution:}}}
    
    \vspace*{0.5cm}

    \begin{center}
        \def\svgwidth{11cm}
        \input{drawing-1.eps_tex}
    \end{center}

    Consider a flat belt drive in which the driven pulley is rotating in the clockwise dirrection.
    
    Let, $T =\  \mbox{Tension in belt on tight side.}$

    $T' =\ \mbox{Tension in the belt on slack side}$

    and, $\theta =\ \mbox{Angle of contact in radian, i.e. Angle suspended by arc AB along which}$ 
    
    the belt touches the pulley at the center.

    Now, consider a small portion PQ of the belt AB, suspending on angle $\delta \theta$ at the center of the pulley.

    The belt PQ is in equilibrium under the following forces,
    \begin{itemize}
        \item Tension $T$ in the belt at P.
        \item Tension $(T + \delta T)$ in the belt at Q.
        \item Normal reaction, R. and,
        \item Frictional force, $F = \mu k$,
        where $\mu$ is the coefficient of friction, between the belt and pulley.
    \end{itemize}

    Resolving all the forces, vertically we have,
    \begin{equation*}
        R = \left(T + \delta T\right)\sin\frac{\delta \theta}{2} + T\sin\frac{\delta \theta}{2} \tag*{(1)}
    \end{equation*}

    Since, the angle $\delta \theta$ is very small, therefore putting $\sin\frac{\delta \theta}{2} = \frac{\delta \theta}{2}$ in equation(1) we get as,
    
    \[R = \left(T + \delta T\right)\cdot \frac{\delta \theta}{2} + T\cdot \frac{\delta \theta}{2}\]

    \[R = T\cdot \frac{\delta \theta}{2} + \delta T \cdot \frac{\delta \theta}{2} + T\cdot \frac{\delta \theta}{2}\]

    \begin{equation*}
        R = T\cdot \delta \theta \tag*{(2)}
    \end{equation*}

    Now, Resolving all the forces horizontally, we have,
    
    \[F = \left(T + \delta T\right)\cos\frac{\delta \theta}{2} - T\cos\frac{\delta \theta}{2}\]

    \begin{equation*}
        \implies\ \mu R = \left(T + \delta T\right)\cos\frac{\delta \theta}{2} - T\cos\frac{\delta \theta}{2} \tag*{(3)}
    \end{equation*}

    Since, the angle $\delta \theta$ is very small, therefore putting $\cos\frac{\delta \theta}{2} = 1$ in equation(3) we get as,

    \[\mu R = \left(T + \delta T\right) - T\]

    \begin{equation*}
        \implies\ R = \frac{\delta T}{\mu} \tag*{(4)}
    \end{equation*}

    equating the value of R from equation(2) and equation(4), we get as,

    \[T\cdot \delta \theta = \frac{\delta T}{\mu} \]
    \[\frac{\delta T}{T} = \mu \cdot \delta \theta\]

    Integrating, both sides, within the limit $T'$ and $T$ and from $0$ to $\theta$ respectively, we get as,

    \[\int_{T'}^{T}\frac{dT}{T} = \int_{0}^{\theta}\mu \cdot d\theta\]
    
    \[\implies\ \log\left(\frac{T}{T'}\right) = \mu \theta\]
    \[\implies\ \frac{T}{T'} = e^{\mu \theta}\]

    \[\implies\ \boxed{T = T'e^{\mu \theta}}\]
    %%Add the diagram

    \begin{center}
        \def\svgwidth{5cm}
        \input{drawing-2.eps_tex}
    \end{center}

    \begin{center}
        from this diagram $\theta = \pi$
    thus, 
    \[\boxed{T = T'e^{\mu \pi}}\]

    Hence, we have the result.
    \end{center}

    \section*{References:}

    For raw data of document, please visit: \url{https://github.com/akhlak919}


    

   
  

   

    

\end{document}        