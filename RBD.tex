\documentclass[a4paper,12pt]{article}
% Set Margins according as you want.
\usepackage{geometry}
\usepackage[utf8]{inputenc}
\usepackage[usenames,dvipsnames]{xcolor}
\pagecolor{white}
\usepackage{graphicx}
\graphicspath{C:\Users\user\Desktop\Learning LaTeX}
\usepackage{amsmath,amssymb}
\usepackage{fancyvrb, fancyheadings}
\usepackage{fancyhdr}
\pagestyle{fancy}
\color{black}
\lhead{\includegraphics[scale=0.12]{DDU_Logo.png}}
\rhead{Rigid Body Dynamics\\Akhlak Ansari}
\cfoot{Page \thepage}
\usepackage{tikz, tcolorbox}
\usepackage{pgfplots}
\pgfplotsset{compat=1.17}
\usepackage{halloweenmath}
\usepackage{enumerate}
\usepackage{tikz, tikz-3dplot}

\title{\underline {\sc {\textbf RIGID BODY DYNAMICS}}}

\author{\textbf{Akhlak Ansari}}

\begin{document}
    \maketitle

    \begin{center}
        \includegraphics[scale=1]{DDU_Logo.png}\\
        {\textbf {\Large Department of Mathematics and Statistics}}
    \end{center}

    \vspace{6cm}

    \begin{center}
        \Large{$\copyright$ All Rights Reserved}
    \end{center}

    \pagebreak
   
    \section*{\underline{Energy Equation:}}

    Let a rigid body be moving with an acceleration $a$. Let an external force $\vec{F}$ acts on an element of mass $m$ of rigid body then the force $\left(-m\vec{a}  + \vec{F}\right)$ acts on this element of the rigid body.Similar forces acting on the other elements of the body,forms a system of forces in equilibrium.

    By the principal of virtual work ,

  
    \begin{equation*}
        \begin{split}
            \vec{F}\cdot d\vec{r} & = 0\\
            \sum\left(-m\vec{a} + \vec{F}\right)\cdot d\vec{r} & = 0\\
            \sum m\vec{a}\cdot d\vec{r} & = \vec{F}\cdot d\vec{ r}\\
            \sum m\vec{a}\cdot \vec{v}dt & = \vec{F}\cdot d\vec{r}\\
            \sum m\frac{d\vec{v}}{dt}\cdot \vec{v}dt & = \vec{F}\cdot d\vec{r}\\
        \end{split}
    \end{equation*}

\end{document}
